

\section*{About This Project}

\paragraph {Abstract:} Gaming today plays an expanding significant role in society. In today's society, gaming is everywhere and is hugely popular among those at a young age, those in their adulthood and even those considered to be over age pensioners are enjoying there time gaming. The reason for this is that there are a wide variety of different genres of games to suit just about any taste that a person may have. From educational purposes that can be both used in schools to teach different topics, like History and Geography to name a few, to those who are simply looking to break away from the shackles of reality by immersing themselves in another world in the shape of a video game. \par
 In recent years the rise of Indie (Independent) Developers in the video game industry has grown exponentially and shows no sign of slowing down as it is becoming easier for just about any person to create their own game with the use of the many gaming engines out there like The Unity Engine, RPG Maker Engine and Unreal Engine 4. An Indie developer is considered either a single person that works on the creation of a game or a small team of two to ten people that work on separate aspects like artwork, sounds, programming and story telling. \par
This project aims to follow the trend of Indie game developers and create something that will allow a form of escape from the stress of the real world that targets any audience, Similar to the thousands upon thousands of small fun indie games that are out there. As the author is in the is in the field of computer science that enjoyed the module of games development, this project will be completed using the knowledge gained from learning about the Unity Game Engine and will appeal to those looking to play a fun mini game that has a one versus one battle scenario. \par 
The project will be a one on one battle game with the objective to reduce the opponents life to zero or to traverse the levels to collect coins in order to gain victory.\par \bigskip

\textbf {Authors:} This project was developed as a 15 credit project by Cian Doyle, A final year student of Software Development at Galway Mayo Institute of Technology.\\

\textbf {Acknowledgements:} The author of this project would like to acknowledge and thank the project supervisors Kevin O'Brien and Dr.John Healy of Galway Mayo Institute of Technology for offering their time and advice throughout this project.
\newpage


\chapter{Introduction}
 During the first three years in GMIT (Galway Mayo Institute of Technology) we had been thought a broad range of topics from hardware to software and several different computing languages including Java, HTML, C, C\# and JavaScript to name just a few. This was done so that we the students could be ready when the time came to choose which path we would like continue in terms of a career. A part of this project is to show what has been learned through the Software development course and to put it in to practice.\par

In 2019 during the first semester of our fourth and final year, we were told that we would be doing a project that would take place over the two semesters. So in October of 2019 I began brainstorming in order to get ideas of what to do in order to start work early. Due to the fact I have always been interested in gaming and entails making your own game, I brought forward an idea to my supervisor of making a small two player versus game. With input from supervisors and from my own ideas, I was able to determine that I would use the Unity game engine to complete my project.\par
In our second semester of third year, we were given a project to do for our Mobile App Development module and this is where I first developed my interest in using the Unity game engine. I created a small endless 3D shooter game in which the player had to fight off a horde of never-ending zombies. The process of seeing my own creation become playable and fun is what really got my interest in games development and in the first semester of our final year I created a small 2D Platformer game for the Mobile Applications Development module again in which we got to experience the developer and customer side of the games development process by being given another students game idea and creating it for them using Unity. Looking at gaming and how much the gaming industry is worth, I wanted to understand it better and have a better grasp on how games are made.

\newpage


\section{Project Objectives}

As mentioned above, the main goals for this project was to gain an understanding of how games are made, the languages used to make them, the different gaming engines that are used for development and to create something light and fun to take a small getaway from the real world to have fun with friends.

This project has been divided into two parts, the research-based dissertation and the applied project. The project will be discussed in terms of the research behind it and the technologies used to build it, this being the reason that it has been split into two parts.\par
The objectives that have been set out for the dissertation are as follows: \par

\textbf {Dissertation:}

\begin{itemize}
  \item \textbf {Introducing the concept of the project:} The reader will be provided with an introduction that will describe the project and will detail its inspiration and goals.\par
  \item \textbf {Provide an understanding of Game engines:} To date, there are a vast amount of technologies available for games development, from Unity, to GameMaker and to UnrealEngine4. Exploring the different technologies I will discuss pros and cons to each and show why I felt my final decision was most suitable for my final project. \par
  \item \textbf {Describe the development of the applied project:} This dissertation aims to give you, the reader, a comprehensive guide into the process of development from the first steps of the initial researching through to the finalized end product. I will describe the approach taken to the applied project, which will include methodologies and technologies used with the design of the games design. For the conclusion I will be discussing any issues that occurred during development, how I went about solving them and what would i have done differently or what I would have liked to of added in future development.\par
\end{itemize}


\textbf {Applied Project:}

\begin{itemize}
  \item \textbf {Make a simple, yet fun to play game:} The project will be able to provide the user with a relatively fun and slightly competitive experience to play with friends or family that can be ran on any computer using Windows. \par
  \item \textbf {Dive into deeper learning into using game engines:} For the purpose of expanding my knowledge and skill-set, the development will be conducted using technologies that have been previously used in the Software Development course. This will further expand on what I have been thought, further progressing the learning outcomes of the modules.\par
  \item \textbf {Complete the project using an efficient and effective approach:} The project will be developed using appropriate tools and methodologies. With the intention of maximising efficiency and effectiveness, a record will be kept of all issues that arose and how they were fixed or why the issues were taken out of the final build.\par
\end{itemize}



\section{Metrics For Success And Failure}

In order for the development of the game to be in a controlled manner along with being able to manage progress in early development, it was vital that metrics were outlined for success or failure. Doing this enabled to keep track of what I wanted to achieve. The simplified list of metrics for the success of the project are as followed:

\begin{itemize}
  \item \textbf {Make a simple, yet fun to play game:} To measure this through different stages of development, I gave my friends outside of the computer science course and fellow students in my year playable versions of levels that I had made in order to gain feedback from different perspectives. \par
  \item \textbf {Make continuous builds to test as the project develops:} Continuously looking for bugs and errors through making build versions of the game to test what, if anything, comes up as an error.
\end{itemize}


\section{An Outline of each chapter}

\paragraph{} This paper has been organised into different chapters with each chapter containing different details in regards to the various aspects of the project. Each of these chapters will be briefly outlined in the following sub-sections.

\subsection{Methodology}
In chapter 2 I will explore what approaches I followed to plan, organize, manage and develop the project. This is where I will discuss the methodologies that were used in order to complete the development of the project along with the research as well as to why they were implemented. The methodology section will provide the reader with an insight as to how the project went from the research to the final product.

\subsection{Technology Review}
In chapter 3 I will cover the technical side of the project which will consist of looking back on the technologies that made up the final version of the project. I will go on to explain explain the different types of technologies that were added and how they were implemented through the project. I will also be explaining why I used the technologies that I included as well as why I decided against using certain technologies as well as the benefits to using one over the other.

\subsection{System Design}
In this chapter I will discuss the architecture and design of the Battle Beans game. This will consist of code snippets and diagrams to help show a basic understanding of the games design. The chapter will begin with a brief overview of the flow the design followed by an in-depth view of what went into making the game and how each feature functions

\subsection{System Evaluation}
This chapter will evaluate the software developed in the project. It will be evaluated in the areas of robustness, testing and scalability as well as a measurement of the results of the final build versus the objectives set out in the beginning. I will also discuss ways i could of improved what was used.

\subsection{Conclusion}
To conclude, a brief review of the overall final project and goals of the project will be outlined. Also to be included is what I would add in further future developments, a final analysis, as well as a review of the discoveries made while researching and the skills learned in the process. For a finish I will leave a brief discussion of my experience working on the project.\cite{}

\section{Requirement Specification}

Below is a list of requirements I made for the project: \par \bigskip
\textbf {Requirements:}

\begin{itemize}
  \item  Be able to load different levels \par
  \item  Be able to determine a winner through score \par
  \item  Allow player to shoot. \par
  \item  Allow Game to end when timer reaches zero \par
  \item  Be able to end game as a draw if scores are equal after the timer reaches zero \par
  \item  Be able to jump through the bottom of a platform \par
  \item  Players can pick up coins to add to score \par
  \item  Players can pick up health drops only if they have lost health \par
  \item  Health items spawn in intervals of 15 seconds  \par
  \item  Coins spawn in intervals of 5 seconds \par
\end{itemize}





